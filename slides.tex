\documentclass{beamer}


\usepackage[utf8]{inputenc}
\usepackage[frenchb]{babel}
\usepackage{verbatim}
\usepackage{graphicx}
\usepackage{color}

\usetheme{boxes}
\usecolortheme{beaver}
\beamertemplatenavigationsymbolsempty
\setbeamertemplate{title page}[default][colsep=-4bp,rounded=true]
\setbeamertemplate{sections/subsections in toc}[circle]
\setbeamertemplate{footline}[frame number]
\setbeamertemplate{itemize items}[circle]
\setbeamertemplate{itemize subitem}[square]

\newcommand{\todo}[1]{\textcolor{red}{[TODO] #1}}

\definecolor{lightgreen}{rgb}{0.0,0.8,0.0}
\definecolor{lightblue}{rgb}{0.3,0.8,1.0}
\definecolor{lightred}{rgb}{0.874,0.180,0.105}
\definecolor{gray}{rgb}{0.4,0.4,0.4}
\definecolor{lightgray}{rgb}{0.8,0.8,0.8}
\definecolor{shadecolor}{rgb}{0.9,0.9,0.9}


\title{Inférence bayésienne adaptative pour la reconstruction de source en dispersion atmosphérique}
\author{Harizo Rajaona}
\institute{CrisTaL - CEA - Aria Technologies}
\date{21 novembre 2016}

\begin{document}
\AtBeginSection[]
{
	\begin{frame}
		\tableofcontents[currentsection]
		% Die Option [pausesections]
	\end{frame}
}
	
	
\begin{frame}
	\titlepage
\end{frame}
\begin{frame}
	\tableofcontents
\end{frame}
\section{Contexte, enjeux et problématique}

\begin{frame}
	(vide)
\end{frame}
\section{Inférence bayésienne et méthodes de Monte-Carlo}
\begin{frame}
	(vide)
\end{frame}
\section{Application au cas expérimental FFT07}
\begin{frame}
	(vide)
\end{frame}
\section{Application avec modèle rétrograde aux cas simulés Beaune et Opéra}
\begin{frame}
	(vide)
\end{frame}
\section{Conclusions et perspectives}
\begin{frame}
	(vide)
\end{frame}
\end{document}
